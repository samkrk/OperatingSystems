\documentclass[12pt, letterpaper]{article}
\usepackage{graphicx}
\usepackage[a4paper, total={6in, 10in}]{geometry}
\usepackage{amsmath}
\usepackage{bm}
\usepackage{listings}
\usepackage{xcolor}
\lstset { %
    language=C++,
    backgroundcolor=\color{black!5}, % set backgroundcolor
    basicstyle=\footnotesize,% basic font setting
}

\title{Operating Systems Notes}
\author{Sam Kirk} 
\date{July 2024}

\begin{document}
\maketitle

\section*{Week 1 (Whole lecture made no sense)}
\subsection*{What is an Operating System?}
\begin{itemize}
    \item A software that converts hardware into a usefull form for applications 
    \item A resource allocator and control program making efficient use 
    of hardware and managing the execution of user programs. 
\end{itemize}
\subsection*{What does an OS provide?}
\begin{itemize}
    \item Abstraction - provides a standard library for resources
    \item (Resources are anything valuable, i.e. CPU, memory, disk)
\end{itemize}
\subsection*{Advantages of Abstraction}
\begin{itemize}
    \item Allows applications to reuse common facilites 
    \item Make different devices look the same 
    \item Provides higher-level or more useful functionality 
\end{itemize}
\subsection*{What is a process?}
\begin{itemize}
    \item An execution stream in the context of a process state 
    \item What is an execution stream?
    \item Stream of executing instructions 
    \item Running piece of code 
    \item Thread of control 
\end{itemize}

\section*{Week 2}
\subsection*{Status Bit}
Determines if we are in user mode or kernel mode. 

\subsection*{Scheduler}
Determines the order that tasks 
are completed. Some common performance 
metrics for the scheduler are:
\begin{itemize}
    \item Turnaround time: completion-time - arrival-time
    \item Response time: initial-schedule-time - arrival-time 
    \item Waiting time: How long tasks spend in the ready queue 
    \item Throughput: Jobs completed per unit of time 
    \item Resource utilization: Keep expensive decives busy 
    \item Overhead: The number of context switches 
    \item Fairness: All jobs get the same amount of CPU over some time interval 
\end{itemize}

Schedulers can be First Come First Served (FCFS), 
Shortest Job First (SJF), Shorted Time to First Completion 
(STCF), RR, Multi Level Feedback Queue (MLFQ),
Completely Fair Scheduler (CFS).

The Dispatcher performs context-switches, i.e. switching from 
user mode to kernel mode. 

In FCFS scheduling, convoys of small tasks tend to built 
up when a large one is running. 

\subsection*{Preemptive Scheduling}
Potentially schedule different jobs at any point 
by taking CPU away from running job. 
So we can run half a task and then if a shorter task comes 
along we can switch and run that task. 





\end{document}