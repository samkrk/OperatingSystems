\documentclass[12pt, letterpaper]{article}
\usepackage{graphicx}
\usepackage[a4paper, total={6in, 10in}]{geometry}
\usepackage{amsmath}
\usepackage{bm}
\usepackage{listings}
\usepackage{xcolor}
\lstset { %
    language=C++,
    backgroundcolor=\color{black!5}, % set backgroundcolor
    basicstyle=\footnotesize,% basic font setting
}

\title{Operating Systems Notes}
\author{Sam Kirk} 
\date{July 2024}

\begin{document}
\maketitle

\section*{Acronym Table}
\begin{itemize}
    \item POSIX: Portable Operating System Interface. a set of standard operating system interfaces based on the Unix operating system
    \item PID: Process Identifier. 
    \item MMU: Memory Management Unit.
    \item PCB: Process Control Block. A structure that contains information about each process
    \item MLFQ: Multi Level Feedback Queue 
    \item CFS: Completely Fair Scheduler
    \item PTE: Page Table Entry 
    \item VPN: Virtual Page Number 
    \item PFN: Page File Number 
    \item DMI: Direct Media Interface 
    \item PCIe: Peripheral Component Interconnect Express
    \item PIO: Programmed IO 
    \item ISR: Interrupt Service Routine 
    \item DMA: Direct Memory Access 
    \item RAID: Redundant Array of Inexpensive Disks
\end{itemize}

\section*{Skipped Textbook Chapters}
\begin{itemize}
    \item 10 
    \item half of 17
    \item Most of 23 
    \item Chapter 27 is confusing 
    \item Chapter 31 is confusing 
    \item 33 
    \item 36 is confusing 
    \item Should probably read 37 again 
\end{itemize}

\section*{Things to work on}
\begin{itemize}
    \item Address translation with paging (chapter 18)
\end{itemize}

\section*{Week 1 (Whole lecture made no sense)}
\subsection*{What is an Operating System?}
\begin{itemize}
    \item A software that converts hardware into a usefull form for applications 
    \item A resource allocator and control program making efficient use 
    of hardware and managing the execution of user programs. 
\end{itemize}
\subsection*{What does an OS provide?}
\begin{itemize}
    \item Abstraction - provides a standard library for resources
    \item (Resources are anything valuable, i.e. CPU, memory, disk)
\end{itemize}
\subsection*{Advantages of Abstraction}
\begin{itemize}
    \item Allows applications to reuse common facilites 
    \item Make different devices look the same 
    \item Provides higher-level or more useful functionality 
\end{itemize}
\subsection*{What is a process?}
\begin{itemize}
    \item An execution stream in the context of a process state 
    \item What is an execution stream?
    \item Stream of executing instructions 
    \item Running piece of code 
    \item Thread of control 
\end{itemize}

\section*{Week 2}
\subsection*{Status Bit}
Determines if we are in user mode or kernel mode. 

\subsection*{Scheduler}
Determines the order that tasks 
are completed. Some common performance 
metrics for the scheduler are:
\begin{itemize}
    \item Turnaround time: completion-time - arrival-time
    \item Response time: initial-schedule-time - arrival-time 
    \item Waiting time: How long tasks spend in the ready queue 
    \item Throughput: Jobs completed per unit of time 
    \item Resource utilization: Keep expensive decives busy 
    \item Overhead: The number of context switches 
    \item Fairness: All jobs get the same amount of CPU over some time interval 
\end{itemize}

Schedulers can be First Come First Served (FCFS), 
Shortest Job First (SJF), Shorted Time to First Completion 
(STCF), RR, Multi Level Feedback Queue (MLFQ),
Completely Fair Scheduler (CFS).

The Dispatcher performs context-switches, i.e. switching from 
user mode to kernel mode. 

In FCFS scheduling, convoys of small tasks tend to built 
up when a large one is running. 

\subsection*{Preemptive Scheduling}
Potentially schedule different jobs at any point 
by taking CPU away from running job. 
So we can run half a task and then if a shorter task comes 
along we can switch and run that task. 

\section*{Prac Assignment 1}
\subsection*{even.c}
\begin{itemize}
    \item fflush(stdout); flushes the output and prints what we have. Without this command it prints everything at the end of the loop. 
    \item SIGINT is an interupt. Triggered by entering ctrl + c. 
    \item SIGHUP is a signal hang-up. Need process PID number to execute. run "kill -SIGHUP (PID number)" in another terminal to interupt.
    \item Get process PID using printf("Process PID: \% d", getpid());
\end{itemize}

\subsection*{minishell.c}
Included libraries and their purpose: 
\begin{itemize}
    \item <sys/types.h> and <sys/wait.h>: For process control functions and types.
    \item <stdio.h>: For input and output functions.
    \item <string.h>: For string manipulation functions.
    \item <unistd.h>: For POSIX operating system API functions.
    \item <stdlib.h>: For general utilities, including memory allocation and process control.
    \item <signal.h>: For signal handling.
\end{itemize}

Questions: 
\begin{itemize}
    \item What is a fork and how does it work?
    \item What is a perror 
    \item What is the execvp function 
    \item How many PIDs can run at once and are they related to the number of cores in the computer 
\end{itemize}


Parent and Child Processes:\\ 
A parent process is a process that creates one or more child processes.
The parent process typically performs operations that involve managing, monitoring, or 
coordinating with its child processes. It may create child processes to perform specific 
tasks concurrently or handle multiple tasks simultaneously.\\ 
A child process is a process that is created by a parent process.
The child process executes instructions independently of the parent process. 
It inherits some properties from the parent process, such as open file descriptors, 
environment variables, and sometimes memory space. However, it has its own unique 
process ID (PID) and may execute different code or perform different tasks.


Possible Return Values from fork()\\ 
\begin{itemize}
    \item If fork() returns a positive integer, this value is the process ID (PID) of the newly created child process
    \item If fork() returns 0, this indicates that the code is executing in the child process
    \item If fork() returns -1, it indicates that an error occurred while attempting to create a new process. 
\end{itemize}

\section*{Week 3}
\subsection*{Managing Processes with Base and Bounds}
Give each process a base register for its memory use and give 
it a bound which it cannot exceed. Protection is required 
so a user cannot change the base and bound registers. 
Also a user cannot change to privileged mode. 
Some advantages of this Base and Bound resgisters are:
\begin{itemize}
    \item Provides memory Protection
    \item Support dynamic relocation for process memories
    \item Simple and inexpensive (few registers, little MMU logic)
    \item Fast (add and compare occur in parallel)
\end{itemize}
Disadvantages are:
\begin{itemize}
    \item Each process has a certian amount of space so this might go unused 
    \item Also this memory is continuous (so its in one big block/list)
    \item Cannot share memory with other processes
\end{itemize}

\subsection*{Paging}
Divide address spaces and physical memory into fixed-size pages. 

\section*{Week 4}

\section*{Week 5}

\section*{Week 6}




\end{document}